\documentclass[a4paper]{article}

\usepackage[utf8]{inputenc}
\usepackage[T1]{fontenc}
\usepackage{textcomp}
\usepackage[french]{babel}
\usepackage{amsmath, amssymb}

\DeclareMathOperator{\E}{E}
\DeclareMathOperator{\V}{V}



\title{Théorie des sondages}

\begin{document}
\maketitle
\tableofcontents

\part{}

\section{}
une fois l'échantillon selectioné on dispose de l'information suivante : $Y_{i1}, Y{i2}, \ldots Y_{in}$. Cherchons toujours teta, il convient de 
combiner ces petites $n$ valeurs recuillies sur l'échantillon pour obtenir une expression dont la valeur numérique est proche de celle de $\theta$. La
formule qui aggrège les petis $n$ valeurs s'appelle l'estimateur de teta, on le note :  $\hat{\theta}$ il s'agit d'une fonction calculée a partir des
données de l'échantillon, et la procédure qui permet de sauter le pas, c'est à dire de passer des odnnées receuillies sur les échantillons à la vrai
valeur inconnue dans la population s'appelle l'inférence statistique (coeur même de la théorie des sondages).

\section{Les mesures des erreurs d'échantillonage}
Une fois que la fonction qui agrège $\hat{\theta} = g(Y_{i1}, Y_{i2}, \ldots Y_{in})$ petit g a été choisie il est nésessaire d'evaluer sa pertinence en
d'autre termes il convient de répondre à la question suivante : étons proche de la valeur $\theta$ lorsque l'on calcule $g$ a partir de
l'échantillon selectionné. En fait il n'est pas possible de répondre précisément à cette question pour la simple et bonne raison que l'on ne connait
pas $\theta$, mais la statistique permet d'apporter des éléments de réponse en exploitant l'aspect probabiliste des choses. Un des points fondamentaux
qu'il faut a présent bien comprendre, réside dans la nature de l'aléa introduit dans notre problème, l'aléa se situe exclusivement au niveau des
identifiants des individus de l'échantillon.
Ce qui est aléatoire sont les $i1, i2, \ldots in$ et non les $ Y_{i1}, Y_{i2}, \ldots Y_{in}$, ainsi si l'on réalise une enquete sur les revenus, le
sondeur considère dans son enquete que chaque individu dans la population est capable de fournir son revenu, en revanche l'échantillon $s$ d'individus
interrogés sur leur revenu aura une compisition aléatoire. Notre estimateur  $g ou \hat{\theta}$ est donc aléatoire fonction de léchantillon $s$. La
réponse à notre question initiale fait appel à trois notions : le biais, la variance, et l'erreur quadratique moyenne. 

Exemple : 
$N=30 n=5$ \\ 
\begin{align*}
    \hat{\overline{Y}} = \overline{y} &\rightarrow s_1 = (2302, \ldots , \ldots ) &\hat{y}(s_1) &= \hat{\overline{Y}} \\
                                      &\rightarrow s_2 = (\ldots, \ldots , \ldots ) &\hat{y}(s_2) &= \hat{\overline{Y}} \\
\end{align*}

\subsubsection{Le biais}
Le biais permet de détecter la présence éventuelle d'erreur systématique, il correspond donc à la différence entre l'espérance mathématique de
l'estimateur  et le paramètre lui même $\text{Biais} = \E(\hat{\theta}) - \theta $. Le biais constitue une premiere mesure d'erreur d'échantillonage.

Exercice :
Considérons une population composée de 4 entreprises $N = 4$ et l'on s'interesse au chiffre d'affaire mensuel moyen de cette population
$\overline{\text{CA}}$ mensuel. Le CA mensuel des entreprises est le suivant :
\begin{itemize}
    \item CA1 = 6000€
    \item CA2 = 12000€
    \item CA3 = 8000€
    \item CA4 = 6000€
\end{itemize}
Supposons que pour raison de contrainte de budget, je ne interroger que 2 entreprises parmis les 4 $n=2$
\begin{equation*}
    \theta = \frac{CA_1 + CA_2 + CA_3+CA_4}{N} = 8000 €  %barré
\end{equation*}
Nombre d'échantillons possibles :
\begin{equation*}
    C^2_4 = \frac{4!}{2!(4-2)!} = 6
\end{equation*}
Ces 6 échantillons sont les suivants : 
\begin{itemize}
    \item $s_1 = (1,2)$
    \item $s_2 = (1,3)$
    \item $s_3 = (1,4)$
    \item $s_4 = (2,3)$
    \item $s_5 = (2,4)$
    \item $s_6 = (3,4)$ 
\end{itemize}
Parce que l'on juge que l'entreprise 1 est particulièrement coopérative sur ce sujet, on veut lui donner une probabilité de tirage supérieure aux
trois autres, si bien que les trois échantillons $s_1, s_2, s_3$ sont un peu plus probables que les autres. La probabilité de selectioner
l'échantillon 1 : 
\begin{itemize}
    \item $P(s_1) = 0.25$
    \item $P(s_2) = 0.25$
    \item $P(s_3) = 0.2$
    \item $P(s_4) = 0.1$
    \item $P(s_5) = 0.1$
    \item $P(s_6) = 0.1$
\end{itemize}
Et par soucis de simplicité on choisi comme estimateur la moyenne simple dans l'échantillon. Autrement dit, on suppose que :
$\hat{\overline{\text{CA}}} = \overline{y}$ La moyenne simple calculée pour les entreprises des échantillons.

\begin{align*}
    &s_1 &\overline{\text{CA}}(s_1) &= \frac{6000 + 12000}{2} = 9000 \\
    &s_2 &\overline{\text{CA}}(s_2) &= \frac{6000 + 8000}{2} = 7000 \\
    &s_3 &\overline{\text{CA}}(s_3) &= \frac{6000 + 6000}{2} = 6000 \\
    &s_4 &\overline{\text{CA}}(s_4) &= \frac{12000 + 8000}{2} = 10000 \\
    &s_5 &\overline{\text{CA}}(s_5) &= \frac{12000 + 6000}{2} = 9000 \\
    &s_6 &\overline{\text{CA}}(s_6) &= \frac{8000 + 6000}{2} = 7000
\end{align*}

Calculons le biais : $\theta = \overline{\text{CA}} = 8000€$
\begin{equation*}
\begin{split}
    \E\left( \theta \right) &= \sum_{s = 1}^{6} P(s) \cdot \hat{\theta}(s) \\
    &= 0.25 * 9000 + 0.25 * 7000 + 0.2 * 6000 + 0.1 * 10000 + 0.1 * 9000 + 0.1 * 7000 = 7800
\end{split}
\end{equation*}
En moyenne l'estimateur est de $7800$, on peut donc calculer le biais : 
\begin{equation*}
   \text{Biais} = 7800 - 8000 = -200 
\end{equation*}
\subsubsection{La variance}
On présent bien au travers de l'exemple précédent que la notion de moyenne ne suffit pas à mesurer la qualité d'un échantillonage et qu'il faut une
autre grandeur d'avantage liée à la dispersion des valeurs de $\hat{\theta}$. On calcule la variance des estimateurs $\hat{\theta}_s$lorsque l'aléa est
l'échantillon, $s$
\begin{equation*}
\begin{split}
\V(\hat{\theta}) &= E \left[ \left( \hat{\theta} _ \E\left( \hat{\theta} \right)  \right)^2  \right] \\
    &= \sum_{s = 1}^6 P(s) \left[ \left( \hat{\theta} _ \E \left( \hat{\theta} \right)  \right)^2 \right]  
\end{split}
\end{equation*}
La présence du terme quadratique fait toute la différence avec le biais puisque les écarts positifs et négatifs ne se compensent plus. En terme de
sondage $\sigma_{\hat{\theta}} \text{et} \V(\hat{\theta})$ mesurent la précision et constituent après le biais une seconde mesure de l'erreur 
d'échantillonage, plus ils sont grands, moins le plan de sondage est bon. Si $\sigma_{\hat{\theta}} \text{et} \V(\hat{\theta})$ sont grands il faut 
agir sur l'expression de $\hat{\theta}$ soit sur les probabilités de tirage ($P(s)$). Le meilleur estimateur $\hat{\theta}$ est celui qui a la plus
petite variance compte tenu du budget dont on dispose.
\begin{equation*}
\begin{split}
    0.25 (9000 - 7800)^2 &= 360000 \\
    0.25 (7000 - 7800)^2 &= 160000 \\
    0.2 (6000 - 7800)^2 &= 648000 \\
    0.1 (10000 - 7800)^2 &= 284000 \\
    0.1 (9000 - 7800)^2 &= 144000\\
    0.1 (7000 - 7800)^2 &= 64000
\end{split}
\end{equation*}

\begin{equation*}
    \V\left( \hat{\theta} \right) = 1860000 
\end{equation*}

Pour terminer, il faut insister une fois encore sur le fait que ni l'espérance, ni la variance ne peuvent renseigner sur l'écart exact entre la valeur
de $\hat{\theta}$ obtenue et la vrai valeur de $\theta$ inconnue.

\subsubsection{L'erreur quadratique moyennei (EQM) (MSE)}
On peut également construire un indicateur de précision qui englobe les notions de biais et de variance pour se faire, il suffit de calculer l'erreur
quadratique moyenne qui est $ \text{EQM = \E\left[\hat{\theta} - \theta \right]^2 } $. Cette erreur quadratique moyenne est un indicateur synthétique
de qualité permettant de répondre à la question complexe suivante : est-il préférable d'avoir un biais fort et une variance faible ou un biais faible
et une variance forte, il n'y a evidement pas de bonne réponse à cette question, mais la pratique montre que l'on cherche en général à éviter en
priorité les biais forts.

Pour conclure : l'erreur d'échantillonage est donc le fait que les résultats numériques publiés à la suite d'un sondage dépendent des individus qui
composent l'échantillon. Elle est présente, et meme une caractéristique des enquetes par sondage. Cette erreur d'échantillonage se mesure par le
biais, la variance ou l'erreur quadratique moyenne, elle ne peut pas être évitée.

\section{Les bases de sondage}
\subsection{Les propriétés des bases de sondages}
Pour pouvoir réaliser un tirage probabilisite, c'est à dire un tirage pour lequel par définition chaque individu de la population a une probabilité
connue et fixée à l'avance de faire partie de l'échantillon à enqueter, il est absolument nécessaire de disposer d'une liste de toutes les unités
d'échantillonage faisant partie du champs de l'enquete, cette liste est appellée base de sondage, et elle doit posseder un certain nombre de qualités.

Pas de double compte
Permet de reperer sans ambiguité les individus
Le moins de défaut de couverture (doit etre exhaustive)



\end{document}
