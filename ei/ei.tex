
\documentclass[11pt,twoside,a4paper]{article}
\usepackage[english]{babel}
\usepackage{a4}
\usepackage[T1]{fontenc}
\usepackage{epsfig}
\usepackage{amsmath, amsthm}
\usepackage{amsfonts,amssymb}
\usepackage{float}
\usepackage{url}
\bibliographystyle{plain}
\title{Analyse financière}
\author{}
\date{}
\begin{document}
\maketitle

\tableofcontents
\newpage

\part{Les états financiers}
\section{Le bilan}
\subsection{Définition}
\textbf{Approche patrimoniale :} le bilan détaille de patrimoine de l'entreprise à une date donnée. Les éléments actifs du patrimoine ont une valeur économique positive pour l'entreprise (biens, créance). Les éléments passifs représentent les éléments du patrimoine ayant une valeur économique négative pour l'entreprise (dettes). Les capitaux propres mesurent la valeur nette du patrimoine. \\ \\
\textbf{Approche économique et financière :} le bilan décrit la situation financière de l'entreprise à une date donnée, il décrit :
\begin{itemize}
    \item L'ensemble des ressources financières que l'entreprise s'est procurée (moyens de financement).
    \item L'ensemble des emplois (utilisation des ressources) dont elle dispose à une date donnée.
\end{itemize}
Les ressources financières sont de trois natures :
\begin{itemize}
    \item Les apports des associés (capital), ces ressources on un caractère permanent.
    \item Les bénéfices, ressources générées par l'activité de l'entreprise.
    \item Les dettes envers les tiers, ressources temporaire.
    \item L'ensemble des ressources constitue le passif du bilan.
\end{itemize}
Parmi les emplois on distingue les emplois permanents (biens durables) et les emplois temporaires liés à l'exploitation. L'ensemble des emplois constitue l'actif du bilan, l'actif égalise le passif, les emplois égalisent les ressources.
\subsection{La structure du bilan : l'actif}
\textbf{L'actif immobilisé :} biens et créances destinées à être utilisés ou à rester de façon durable dans l'entreprise, il est composé de trois postes : 
\begin{itemize}
    \item immobilisations corporelles
    \item immobilisation financière
    \item immobilisation incorporelle
\end{itemize}
\textbf{L'actif circulant :} biens et créances liés à l'exploitation et qui n'ont pas vocation à être maintenus durablement dans l'entreprise. L'actif circulant est composé de :
\begin{itemize}
    \item Stock \& En-cours
    \item Créances
    \item Valeurs mobilières de placement
    \item Les disponibilités
\end{itemize}
\subsection{La structure du bilan : l'actif}
\textbf{Les capitaux propres :}  moyens de financement mis à la disposition de l'entreprise de façon permanente (ressource interne de financement). \\ \\
Ils comprennent deux principaux postes, \textbf{le capital de réserves :} les apports des associés et la part de bénéfice non distribué, laissée à la disposition de l'entreprise, ainsi que \textbf{du résultat de l'exercice :} bénéfice ou perte.\\ \\
\textbf{Les dettes financière :} moyens de financement externe (ressources externes de financement). \\ \\ 
Les dettes comprennent comme principaux postes, \textbf{les dettes financière :} les emprunts effectué auprès de l'établissement de crédit, \textbf{les dettes d'exploitation :} moyens de financement liés au cycle d'exploitation (dettes fournisseurs, fiscales et sociales), \textbf{les autres dettes :} les dettes non liées au cycle d'exploitation (dettes fournisseurs d'immobilisation, etc.). 
\subsection{Présentation schématique du bilan}

\end{document}
