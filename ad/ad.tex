\documentclass[a4paper]{article}

\usepackage[utf8]{inputenc}
\usepackage[T1]{fontenc}
%\usepackage{textcomp}
\usepackage[french]{babel}
\usepackage{amsmath, amssymb}

\usepackage[
    left = \flqq{},% 
    right = \frqq{},% 
    leftsub = \flq{},% 
    rightsub = \frq{} %
]{dirtytalk}

\usepackage{nicematrix}


\setlength{\parindent}{0em}

\title{Analyse des données}
\begin{document}

\maketitle
\tableofcontents
\newpage

\section{Introduction}
Les méthodes statistiques sont utilisées dans presque tous les secteurs de l'activité humaine et font parti des connaissances de base de
l'ingénieur, du gestionnaire, de l'économiste, du chercheur \ldots \\ 
A l'issue de la phase de recueil de données, la démarche statistique consiste à traiter et interpreter les informations recueillies. Cette démarche
comporte souvant deux grands aspects, d'abord l'aspect descriptif ou exploratoire et dans un second temps l'aspect inférentiel ou décisionnel. \\
\begin{itemize}
    \item[] \textbf{La statistique exploratoire} a pour objet de synthetiser, résumer et structurer l'information contenue dans les données et utilise
        pour cela des représentations des données sous forme de tableau, graphiques et ou indicateurs.
    \item[] \textbf{La phase décisionnelle} arrive après la phase exploratoire, mais nous n'allons pas la traiter \\
\end{itemize}
Depuis plus de 50 ans les méthodes d'analyse de données ont largement démontré leur efficacité dans l'étude de gros volumes de donnés grace a
l'informatique. \\
Les méthodes dites \textbf{multi-dimensionelles} telles que l'analyse en composante principale (ACP) ou l'analyse factorielle des correspondances (AFC)
permettent la mise en relation de nombreuses variables entre elles.\\
Les méthodes multi-dimensionelles permettent d'obtenir des représentations graphiques qui constituent le meilleur résumé possible de l'information dans 
un gros tableau de données.\\
Pour cela le statiticien conscent à une perte d'information afin de gagner en lisibilité, il est alors en mesure de faire apparaitre les principaux
phénomènes qu'il cherche à analyser.\\

Il est possible de diviser les principales méthodes d'analyse de données en deux grands groupes : \\

\begin{itemize}
    \item[] \textbf{Les méthodes de classification} qui ont pour objet de réduire la taille de l'ensemble des individus en formant des groupes homogènes
    \item[] \textbf{Les méthodes factorielles} cherchent à réduire le nombre de variables en les résumant par un petit nombre de composants synthétiques.
        Selon que les variables soient quantitives ou qualitatives, on utilisera soit l'analyse en composante principale, soit l'analyse factorielle des
        correspondances. \\
\end{itemize}
Le choix d'une méthode statistique dépend de l'objectif que l'on se fixe, mais également de la nature des variables, ainsi que le nombre de variables.

\subsection{Les différents types de variables}
Une variables statistique décrit une caractéristique pour les différents individus. On distingue deux grands types de variables, les variables
qualitatives et quantitatives.
\begin{itemize}
    \item Les variables quantitatives qualifient des quantités, elles peuvent se classer en variables continues ou discrètes.
    \item Les variables qualitatives définissent une caractérisitque (des qualités), elles peuvent être de deux natures : nominales ou ordinales.
\end{itemize}
\subsection{Description d'une variable quantitative}
Une variable quantitative est décrite par l'ensemble des valeurs qu'elle prend pour n individus. Afin de synthétiser l'information, on peut utiliser la
moyenne ou la variance.
\begin{align*}
    \overline{x} &= \frac{1}{n}\sum_{i = 1}^{n} x_i & V(x) = \frac{1}{n} \sum_{i = 1}^{n} \left( x_i - \overline{x} \right)^2
\end{align*}
Par définition une variable est centrée si sa moyenne est nulle et est réduite et, si sa variance est égale à 1. L'interet de centrer et réduire permet de
donner une interprétation géométrique au coefficient de correlation linéaire.
\subsection{Description d'une variable qualitative}
Pour décrire les variables qualitatives, on utilise des fréquences relatives et les tableaux disjonctifs complets.\\

Considérions la variable qualitative \say{couleur des yeux} avec trois modalités : bleu, vert et marron.
\begin{center}
\begin{tabular}{|c|c|c|c|c|c|c|}
\hline
Individu         & 1    & 2    & 3      & 4    & 5    & 6    \\ \hline
Couleur des Yeux & Bleu & Vert & Marron & Vert & Bleu & Bleu \\ \hline
\end{tabular}
\end{center}
L'effectif est le nombre d'individus ayant une modalité dans l'échantillon. On peut donc a partir de l'effectif, calculer la fréquence relative
qui est l'effectif divisé par le nombre total d'individus. \\

La présentation d'une variable qualitative sous sa forme disjonctive complète est celle qui se prête le mieux à des calculs statistiques. Cette
dernière s'obtient en définissant une variable indicatrice pour chacune des modalités de la variable (\say{B} pour bleu \ldots). \\

Tableau disjonctif complet des données ci-dessus : 
\begin{equation*}
\text{X}=
\begin{bNiceMatrix}[first-row]
    \text{B} & \text{V} & \text{M} \\ 1 & 0 & 0 \\ 0 & 1 & 0 \\ 0 & 0 & 1 \\ 0 & 1 & 0 \\ 1 & 0 & 0 \\ 1 & 0 & 0
\end{bNiceMatrix}
\end{equation*}
Propriétés du tableau disjonctif :
\begin{itemize}
    \item La somme des colonnes du tableau complet est égale à un vecteur colonne de dimension $n$ dont tous les éléments sont égaux à 1. Chaque
        individu possède une seule modalité, cela signifie que sur une ligne de données ne figure que des 0, à l'exception d'un élément unique
        égal à 1.
    \item Le produit matriciel de la transposée de la matrice X fois X est une matrice diagonale dont les éléments sont les effectifs de
        chacune des modalités. Exemple avec les données précédentes : \\
\end{itemize}
\begin{equation*}
    \text{X'X} = 
    \begin{bmatrix} 1 & 0 & 0 & 0 & 1 & 1 \\ 0 & 1 & 0 & 1 & 0 & 0 \\ 0 & 0 & 1 & 0 & 0 & 0 \end{bmatrix} 
    \cdot
\begin{bNiceMatrix} 1 & 0 & 0 \\ 0 & 1 & 0 \\ 0 & 0 & 1 \\ 0 & 1 & 0 \\ 1 & 0 & 0 \\ 1 & 0 & 0 \end{bNiceMatrix}
= \begin{bmatrix} 3 & 0 & 0 \\ 0 & 2 & 0 \\ 0 & 0 & 1\end{bmatrix} 
\end{equation*}



\subsection{Relation entre deux variables quantitatives}
Une variable $x$ prenant $n$ valeurs peut être représentée par un vecteur dans un ensemble $\mathbb{R}^n$. Dans cet espace, le produit scalaire
entre deux vecteurs $\vec{x}$ et $\vec{y}$, est égal à la
somme pour l'ensembles des individus de $x$ et $y$ 
\begin{equation*}
    \vec{x} \cdot \vec{y} = \sum_{i = 1}^n x_i y_i 
\end{equation*}
En statistiques, le produit scalaire utilisé est :
\begin{equation*}
    \vec{x} \cdot \vec{y} = \frac{1}{n}\sum_{i = 1}^n x_i y_i 
\end{equation*}
Ce produit scalaire permet de donner une interpréation géométrique du coefficient de corrélation linéaire. Le cosinus de l'angle formé entre les deux
variables est égal au coefficient de correlation entre ces deux variables tel que : $\cos(\text{X,Y}) = r_{\text{X,Y}}$. \\

Si le coefficient de corrélation est égal a 1, les deux vecteurs sont colinéaires (les valeurs prises par $x_i$ et $y_i$ sont proportionelles).
L'abscence de correlation se traduit par un coefficient de corrélation nul et par un angle droit entre $x$ et $y$.
\subsection{Relation entre une variable quantitative expliquée et un ensemble de variables quantitatives explicatives}

\end{document}
